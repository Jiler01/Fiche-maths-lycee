\documentclass{article}
\usepackage{physics}
\usepackage{amsmath}
\usepackage{amsfonts}
\usepackage[T1]{fontenc}
\usepackage[french]{babel}
\usepackage{enumitem}
\usepackage{hyperref}
\setlength{\parindent}{0pt}

\newlist{impl}{itemize}{1}
\setlist[impl]{label=$\Leftrightarrow$}

\title{Mathématiques lycée}
\date{2022\textemdash 2024}

\begin{document}

\maketitle
Ce livret est un récapitulatif de mes années de spécialités mathématiques de première et terminale et de maths expertes au Lycée International de Ferney-Voltaire durant les années 2022\textemdash2024.\\
Remerciements spéciaux à Mme Escot, M Tierno, Mme Beltzung, Mme Leclerc et feu M Colin  pour leurs connaissances, soutiens et passions, ainsi qu'a Yvan Monka (\href{https://www.maths-et-tiques.fr/}{\textit{Maths et tiques}}) et Grant Sanderson (\href{https://www.3blue1brown.com}{\textit{3Blue1Brown}}), pour leur contenu de la plus haute qualité\\
\break
\underline{Notation:}\\
Pour des raisons de simplicité, ce livret utilise des notations correctes mais avec lesquelles vous pourriez ne pas être familier-re. En voici donc quelques unes expliquées:
\begin{itemize}
	\item $\exp(x) = \sum_{k=0}^{\infty} \frac{x^k}{k!}$ (\href{https://youtu.be/3d6DsjIBzJ4}{\underline{\textit{explication (anglais)})}}
	\item $x^k$ ($x\in\mathbb{R^{+*}}$) équivaut à $\exp(k\ln(x))$
	\item $x^\frac{1}{n}$ équivaut à $\sqrt[k]{x}$
	\item On note $f(g(x) = (f\circ g)(x)$ avec $f$ et $g$ des fonctions.
	\item Soit $f$ une fonction, on note $f'$ sa dérivée, $f''$ sa dérivée seconde, $f^{(n)}$ sa dérivée n-ième, et $F$ sa primitive.
	\item Soit $\mathbb{I}$ un intervalle, $\mathbb{I^+}$ est l'ensemble des éléments positifs de $\mathbb{I}$, et de même pour $\mathbb{I^-}$. De plus, l'ensemble $\mathbb{I^*}$ équivaut à l'ensemble $\mathbb{I}$ privé de son élément nul.
	\item $\forall$ signifie "pour tout"
	\item $\in$ signifie "dans"
	\item $\exists$ signifie "il existe", $\exists!$ signifie "il existe un unique"
	\item Soit $f$ une fonction, $[f(x)]_a^b = f(b)-f(a)$
	\item Soit $f$ une fonction, on note $\mathbb{D}_f$ son ensemble de définition 
\end{itemize}

\pagebreak

\tableofcontents

\pagebreak

\pagebreak\part{Algèbre}
\setcounter{section}{0}
\renewcommand*{\theHsection}{chX.\the\value{section}}
\section{Fonctions}


\subsection{Propriétés de fonction}

\subsubsection{Croissance \& décroissance}
\underline{Pour tout $a<b$ tels que $a,b\in\mathbb{I}$:}\\
Si $f(a) < f(b)$: la fonction est \emph{strictement croissante} sur $\mathbb{I}$\\
Si $f(a) \le f(b)$: la fonction est \emph{croissante} sur $\mathbb{I}$\\
Si $f(a) \ge f(b)$: la fonction est \emph{décroissante} sur $\mathbb{I}$\\
Si $f(a) > f(b)$: la fonction est \emph{strictement décroissante} sur $\mathbb{I}$\\

\subsubsection{Parité \& imparité}
\underline{$\forall x \in \mathbb{D}_f$ :}\\
Si $f(-x) = f(x)$: la fonction est \emph{paire}\\
Si $f(-x) = -f(x)$: la fonction est \emph{impaire}

\subsubsection{Périodicité}
\underline{$\forall x \in \mathbb{D}_f$ :}\\
Si $f(x + k) = f(x)$: la fonction est \emph{k-périodique}
\subsection{Fonctions usuelles}
\subsubsection{Fonctions communes}
\paragraph{Fonctions affines}
Une fonction est une fonction de style $f(x) = ax+b$ avec $a$ la dérivée de la fonction, et $b$ son ordonnée à l'origine.


\paragraph{Fonction carrée}
La fonction carrée est définie par $f(x) = x^2$. Elle est paire sur $\mathbb{R}$, décroissante sur $\mathbb{R^-}$ et croissante sur sur $\mathbb{R^+}$.


\paragraph{Fonction cube}
La fonction cube est définie par $f(x) = x^3$. Elle est impaire et croissante sur $\mathbb{R}$.


\paragraph{Fonction inverse}
La fonction inverse est définie par $f(x) = \frac{1}{x}$. Elle est impaire et décroissante sur $\mathbb{R^*}$.


\paragraph{Fonction racine carrée}
La fonction carrée est définie sur $\mathbb{R^+}$ par $f(x) = \sqrt{x}$. Elle est croissante.

\subsubsection{Fonction valeur absolue}
La fonction valeur absolue est définie par $f(x) = \abs{x} $. Elle est paire sur sur $\mathbb{R}$, décroissante sur $\mathbb{R^-}$ et croissante sur sur $\mathbb{R^+}$.\\
Son comportement suit la règle suivante : $\abs{x} = \abs{-x} = x$.

\subsubsection{Logarithme népérien}
La fonction logarithme népérien, notée $\ln$ est la fonction telle que $\forall x > 0, \ln(x) = y \mid \exp(y) = x$.
\paragraph{Propriétés}
\begin{enumerate}
	\item Pour tout $x>0$,  $\exp(\ln(x)) = x$
	\item Pour tout $x \in \mathbb{R}$, $\ln(\exp(x)) = x$
	\item La fonction logarithme népérien est strictement croissante et concave.
	\item $\ln(a) = \ln(b) \Leftrightarrow a=b$
	\item $\ln(a\times b) = \ln(a) + \ln(b)$
	\item $\lim \limits_{x \rightarrow +\infty} \ln(x) = +\infty$
	\item $\lim \limits_{x \rightarrow 0^+}\ln(x) = -\infty$
	\item Pour tout $x>0$, $(\ln ')(x) = 1/x$
\end{enumerate}


\subsubsection{Fonctions trigonométrique }

\paragraph{Cercle trigonométrique}
Le cercle trigonométrique est défini par un cercle de rayon $1$, donc de périmètre $2\pi$.\\
On admettra que son centre $O$ est aux coordonnées $(0;0)$.

\paragraph{Radian}
Le radian est l'unité de mesure d'angle correspondant à une portion du cercle-unité (de rayon 1 ua) telle que l'arc du cercle associé à cette portion a pour longueur 1 ua. Noté $rad$, on retiendra que $1^\circ= \frac{\pi}{180} rad$ .

\paragraph{Points sur l'arc}
Le point correspondant a l'angle $x$ radians, aura pour coordonnées $(\cos(x);\sin(x))$. \\De plus, l'angle $x$ est le même que l'angle $x +2k\pi$, $\forall k \in \mathbb{Z}$.

\paragraph{Fonctions cosinus et sinus}
\subparagraph{Cosinus}
$\cos$ est une fonction paire, variant entre $-1$ et $1$, et de période $2\pi$, tel que $\cos(x + 2k\pi) = \cos(x)$, $\forall k \in \mathbb{Z}$.\\
\subparagraph{Sinus}
$\sin$ est une fonction impaire, variant entre $-1$ et $1$, et de période $2\pi$, tel que $\sin(x + 2k\pi) = \sin(x)$, $\forall k \in \mathbb{Z}$.


\subsubsection{Fonctions polynômes du second degré}

\paragraph{Formes}
\begin{enumerate}
	\item Forme développée (ou forme développée réduite : $f(x) = ax^2+bx+c$ avec $a \neq 0$
	\item Forme factorisée : $f(x) = a(x - x_1)(x - x_2)$ avec $x_1$ et $x_2$ les racines de $f$\\
	      On notera que $x_1+x_2=-b/a$ et que $x_1x_2=c/a$
	\item Forme canonique: $f(x) = a(x-\alpha)+\beta$ avec $\alpha = \frac{-b}{2a}$ et $\beta = f(\alpha)$\\
	      On notera que $(\alpha;\beta)$ est l'extremum de la fonction, que l'on peut retrouver en étudiant sa dérivée.
\end{enumerate}

\paragraph{Variations}
Pour $a > 0$: décroissante avant $\alpha$, avec un minimum en $\beta$, puis croissante.\\
Pour $a < 0$: croissante avant $\alpha$, avec un maximum en $\beta$, puis décroissante.

\paragraph{Résolution}
\begin{enumerate}
	\item On calcule le discriminant tel que $\Delta = b^2 - 4ac$.
	\item On calcule les racines telles que $x_0 = \frac{-b \pm \sqrt{\Delta}}{2a}$ si $\Delta \ge 0$, $\frac{-b \pm i\sqrt{\abs{\Delta}}}{2a}$ sinon.
\end{enumerate}

\paragraph{Signe de la fonction}
\begin{enumerate}
	\item Si $\Delta > 0$: Du signe de $a$ à l'extérieur des racines, de l'autre entre elles. S'annule sur ces dernières
	\item Si $\Delta = 0$: Du signe de $a$ sur $\mathbb{R}$ S'annule sur la racines.
	\item Si $\Delta < 0$ Du signe de $a$ sur $\mathbb{R}$.
\end{enumerate}



\section{Suites}


Une suite numérique $u$ est une suite de valeurs pourvues d'un indice, $n$, qui parcours l'ensemble $\mathbb{N}$. L'élément à l'indice $n$ de la suite $u$ est noté $u_n$.


\subsection{Modes de génération}

\subsubsection{Liste}
Une simple liste de nombres. (ex: les décimales de $\pi$: 1; 4; 1; 5\dots)

\subsubsection{Formule explicite}
Définir $u_n$ en fonction de $n$: $u_n = f(n)$

\subsubsection{Récurrence}
Définir $u_n$ en fonction de un ou plusieurs $u_{n-k}$ (ex: suite de Fibonacci: $u_n = u_{n-2} + u_{n-1}$)


\subsection{Quelques styles de suites}

\subsubsection{Suite arithmétique}
Une suite arithmétique est définie par $u_n = u_{n-1} + r$ ou $u_n = u_0 + nr$ avec $r$ sa raison.\\
La somme de ses termes d'indices de $p$ à $n$ vaut $(n+1-p)\times \frac{u_p+u_n}{2}$.\\
Pour prouver qu'une suite est arithmétique, il suffit que $r$ dans $u_n - u_{n-1} = r$ soit indépendant de $n$.

\subsubsection{Suite géométrique}
Une suite arithmétique est définie par $u_n = u_{n-1} \times q$ ou $u_n = u_0 \times q^n$ avec $q$ sa raison.\\
La somme de ses termes d'indices de $p$ à $n$ vaut $u_p\times \frac{1-q^{n-p+1}}{1-q}$.\\
Pour prouver qu'une suite est géométrique, il suffit que $q$ dans $\frac{u_n}{u_{n-1}} = q$ soit indépendant de $n$.


\subsection{Croissance et décroissance}
Pour prouver la croissance ou la décroissance d'une suite, on applique \\$u_{n-+1} - u_n$, qui si négatif prouve une décroissance, et inversement.



\section{Polynômes}
Le terme polynôme $P$ de degré $n$ à coefficients réels est une expressions qui s'écrit sous la forme $$P(z) = \sum^n_{i=0}c_iz^i \mid \forall i \in \mathbb{N}, c_i\in \mathbb{R} \text{ et } c_n \neq 0 $$.
On nomme $a$ une racine de $P$ ssi $P(a) = 0$
\subsection{Nombre de racines}
Le nombre de racine d'un polynôme est inférieur ou égal à son degré.\\

\pagebreak\part{Analyse}
\setcounter{section}{0}
\renewcommand*{\theHsection}{chX.\the\value{section}}
\section{Limites}
\href{https://youtu.be/kfF40MiS7zA}{\underline{\textit{Vidéo (anglais)}}}\\
\break
Les règles suivantes peuvent s'appliquer aux fonctions comme aux suites, qui ne sont rien d'autre que des fonctions ne prenant pour paramètre que des entiers naturels.
\begin{enumerate}
	\item $v_n$ admet une limite $+\infty$ si $\forall A \in \mathbb{R}$, il existe un rang $n_0$ tel que $\forall n \ge n_0$, $u_n > A$
	\item $v_n$ admet une limite $-\infty$ si $\forall A \in \mathbb{R}$, il existe un rang $n_0$ tel que $\forall n \ge n_0$, $u_n < A$
	\item $v_n$ admet une limite $l$ si $\forall E \in \mathbb{R^{+*}}$, il existe un rang $n_0$ tel que $\forall n \ge n_0\,l-E<u_n<l+E$
\end{enumerate}
Note: $\lim \limits_{n \rightarrow a^+}$ signifie la limite en $a$ par le coté positif, et inversement pour $\lim \limits_{n \rightarrow a^-}$

\subsection{Opérations}
Lors du calcul d'une limite, les formes indéterminées sont:
\\$$\infty - \infty \qquad 0 \times \infty \qquad \frac{0}{0} \qquad \frac{\infty}{\infty}$$
Si on en rencontre une, l'usage est de manipuler l'expression de sorte à ne plus avoir ce problème, ou bien d'appliquer la règle de L'Hôpital selon laquelle $$\lim \limits_{x \rightarrow a} \frac{f(x)}{g(x)} = \lim \limits_{x \rightarrow a} \frac{f'(x)}{g'(x)}$$ ssi $a$ est un réel ou $+\infty$ ou $-\infty$, $\lim \limits_{x \rightarrow a} \frac{f(x)}{g(x)} = \frac{0}{0}$ ou $\frac{\pm\infty}{\pm\infty}$, $f'(x)$ et $g'(x)$ existent, et $g'(x) \ne 0$.

\subsection{Comparaison et encadrement}
On suppose deux suites $u_n$ et $v_n$ telles que après un certain rang, $u_n>v_n$.
$$\lim \limits_{n \rightarrow \infty} v_n = + \infty \implies \lim \limits_{n \rightarrow \infty} u_n = + \infty$$
$$\lim \limits_{n \rightarrow \infty} u_n = - \infty \implies \lim \limits_{n \rightarrow \infty} v_n = - \infty$$
On suppose $u_n$, $v_n$ et $w_n$, telles que après un certain rang, $u_n \le v_n \le w_n$
$$\lim \limits_{n \rightarrow \infty} u_n = \lim \limits_{n \rightarrow \infty} w_n = l \implies \lim \limits_{n \rightarrow \infty} v_n = l$$

\subsection{Notion d'asymptote}
Soit $f$ une fonction et $a$ et $l$ des réels.
\begin{itemize}
	\item $\lim \limits_{n \rightarrow \pm \infty} f = l \implies x=l$ est une asymptote horizontale à la courbe $f$
	\item $\lim \limits_{n \rightarrow a} f = \pm \infty \implies y=a$ est une asymptote verticale à la courbe $f$
\end{itemize}

\subsection{Croissance comparée}
Soit $x\in\mathbb{R}$ et $n\in\mathbb{N}$
On a $\exp(x)$ qui crois et décroit plus vite que $x^n$.\\
Ainsi, $\lim \limits_{n \rightarrow \infty} \frac{\exp{x}}{x^n} = \infty$ et $\lim \limits_{n \rightarrow -\infty} \exp{x}\times x^n = 0$.


\section{Dérivation}
\subsection{Dérivation locale}

\subsubsection{Taux de variation}
Le taux de variation entre $a$ et $b$, soit la croissance moyenne entre $a$ et $b$, peut être mesuré tel que : $\frac{f(b) - f(a)}{b - a}$.\\

\subsubsection{Nombre dérivé (dérivée)}
La dérivée de $f(x)$, soit la variation de $f$ en $x$, s'écrit $f'(x)$. Il s'agit du taux de variation instantané, alors que le deuxième point se confond avec le premier. Sa formule est donc:
$$f'(x) = \lim \limits_{h \Rightarrow 0} \frac{f(a + h) - f(a)}{h}$$

\subsubsection{Tangente à une courbe}
La tangente à la courbe de $f$ en $a$ est la droite définie par: $$\mathcal{T} = f'(a)(x-a) + f(a)$$


\subsection{Dérivation globale}
\href{https://youtu.be/9vKqVkMQHKk}{\underline{\textit{Vidéo (anglais))}}}\\
\href{https://youtu.be/S0_qX4VJhMQ}{\underline{\textit{Approche géométrique}}}\\
\href{https://youtu.be/CfW845LNObM}{\underline{\textit{Approche non-conventionnelle}}}
\break
Une dérivation globale reprend le concept de dérivation locale, mais fourni une formule applicable sur toute la fonction. Elle est utilisée pour étudier les variations de la fonction.

\subsubsection{Dérivées de fonctions usuelles}
$f(x) \rightarrow f'(x)$ avec $f$ une fonction et $k\in\mathbb{R}$
\begin{enumerate}
	\item[] $$k \rightarrow 0$$
	\item[] $$kx \rightarrow k$$
	\item[] $$x^n \, (n\in\mathbb{R}, x\ne 0 \text{ si } n\le 1) \rightarrow nx^{n-1}$$
	\item[] $$e^x \rightarrow e^x$$
	\item[] $$\abs{x} \rightarrow \frac{x}{\abs{x}}$$
	\item[] \begin{center}Pour les fonctions trigonométriques, on tourne dans le sens horaire.\end{center}
\end{enumerate}

\subsubsection{Opérations sur les dérivées}
\begin{enumerate}
	\item[] $$\forall k \in \mathbb{R} : (ku)' = ku'$$
	\item[] $$(u+v)' = u' +v'$$
	\item[] $$(uv)' = vu' + uv'$$
	\item[] $$(\frac{u}{v})' = \frac{u'v-v'u}{v^2} $$
	\item[] $$u\circ v \rightarrow v'  \times u'\circ v$$
\end{enumerate}

\subsubsection{Variations de la fonction}
$\forall f$, si $f'$ est négatif, $f$ est décroissante. Inversement, si $f'$ est positive, $f$ est croissante.


\subsection{Dérivée seconde}
\href{https://youtu.be/BLkz5LGWihw}{\underline{\textit{Vidéo sur les dérivées d'ordre supérieur (anglais)}}}
La dérivée seconde d'une fonction est la dérivée de la dérivée de cette fonction. Elle représente l'évolution de l'évolution de la fonction.

\subsubsection{Convexité}
Une fonction est convexe quand elle est au dessus de ses fonctions, concave si elle est en dessous. Le point ou elle change de convexité est son point d'infection.\\
On notera que $f''(x)>0 \mid x\in \mathbb{I} \Rightarrow f$ convexe sur $\mathbb{I}$, $f''(x)<0 \mid x\in \mathbb{I} \Rightarrow f$ concave sur $\mathbb{I}$.

\section{Continuité}
\begin{enumerate}
	\item Une fonction $f$ est dite continue au point $a$ si $\lim \limits_{x \rightarrow a} f(x) = f(a)$ (et que donc ces deux éléments existent).
	\item Une fonction est continue sur l'intervalle $\mathbb{I}$ si elle est continue en tout point de $\mathbb{I}$.
\end{enumerate}
\subsection{Propriétés}
\begin{enumerate}
	\item Si une fonction $f$ est dérivable en un point $a$ alors $f$ est continue en $a$.
	\item Les fonctions usuelles (fonction carrée, fonction cube, fonction inverse, fonction exponentielle, fonction sinus, fonction cosinus et les fonctions polynômes) sont continues sur leur ensemble de définition.
	\item Tout fonction définie comme la somme, le produit, le quotient ou la composée de fonctions continues est continue sur son ensemble de définition.
\end{enumerate}
\subsection{Théorème des valeurs intermédiaire}
Si la fonction $f$ est définie et continue sur un intervalle $[a;b]$, alors $$\forall k \in [f(a);f(b)], \exists x \in [a;b] \mid f(x) = k$$
\subsection{Théorème de la bijection}
Si la fonction $f$ est définie, continue et strictement monotone sur un intervalle $[a;b]$, alors $$\forall k \in [f(a);f(b)], \exists ! x \in [a;b] \mid f(x) = k$$
\subsection{Théorème du point fixe (suites)}
Soit $f$ une fonction et $u$ une suite telle que $u_0 \in \mathbb{R}$ et $u_{n+1} = f(u_n)$, alors si $f$ est continue et $\lim \limits_{n \rightarrow +\infty} u_n \in \mathbb{R}$, $l$ est solution de l'équation $l = f(l)$.


\section{Primitive}
Une primitive est l'opposée d'une dérivée. Soit $f$ une fonction, la primitive de $f$ est la fonction $F$ telle que $F' = f$.

\subsection{Primitives de fonctions usuelles}
$f(x) \rightarrow F(x)$ avec $f$ une fonction et $k,c\in\mathbb{R}$.
\begin{enumerate}
	\item[] $$k \rightarrow kx + c$$
	\item[] $$x^n \, (n\in \mathbb{R} \setminus \{-1\}) \rightarrow \frac{x^{n+1}}{n+1}+c$$
	\item[] $$\frac{1}{x} \, (x>0) \rightarrow \ln(x)+c$$
	\item[] $$e^x \rightarrow e^x+c$$
	\item[] \begin{center}Pour les fonctions trigonométriques, on tourne dans le sens anti-horaire.\end{center}
\end{enumerate}

\subsection{Opérations sur les primitives}
$f(x) \rightarrow F(x)$ avec $u$ et $v$ des fonction, $U$ et $V$ leurs primitives respectives et $k\in\mathbb{R}$.
\begin{enumerate}
	\item[] $$\forall k \in \mathbb{R} : (ku)' = kU$$
	\item[] $$u+v \rightarrow U + V$$
	\item[] $$Vu + Uv \rightarrow UV$$
	\item[] $$\frac{Uv-Vu}{V^2} \rightarrow \frac{u}{v}$$
	\item[] $$V \times U\circ v \rightarrow u\circ v$$
\end{enumerate}

\subsection{Ensemble de primitives}
Soit $F$ une primitive de $f$, $G: x\rightarrow F(x) + k\,k\in\mathbb{R}$, est aussi une primitive de $f$, ce qui peut facilement se vérifier en la dérivant. Ainsi $\{H: x\rightarrow F(x) + k \forall k\in\mathbb{R}\}$ est l'ensemble des primitives de $f$.
\subsection{Primitive particulière}
On ajoute que si $f$ est continue sur $\mathbb{I}$, $\forall x\in\mathbb{I}$ et $y\in\mathbb{R}$, $\exists ! F\mid F'=f$ et $F(x)=y$.

\section{Équations différentielles}
\subsection{De type $y'=ay$}
Les solution au équations de type $y'=ay$ avec $y$ une fonction, sont les fonctions $y$ telles que $y(x) = Ce^{ax}$ avec $C\in\mathbb{R}$
\subsection{De type $y'=ay+b$}.
Les solution au équations de type $y'=ay+b$ avec $y$ une fonction et $b\in\mathbb{R}$, sont les fonctions $y$ telles que $y(x) = Ce^{ax}-\frac{b}{a}$ avec $C\in\mathbb{R}$
\subsection{De type $y'=ay + f$}.
Les solution au équations de type $y'=ay +f$ avec $y$ et $f$ des fonctions, sont les fonctions $y$ telles que $y(x) = Ce^{ax} + u(x)$ avec $C\in\mathbb{R}$ et $u$ une solution particulière de l'équation.

\section{Intégration}
L'intégrale de la fonction $f$ entre les bornes $a$ et $b$ et l' "aire signée" (négative sous $y=0$, positive au dessus) entre la courbe de $y=f(x)$ et la droite $y=0$, entre les droites $x=a$ et $x=b$. Elle est notée $\int_{a}^{b} f(x) \,dx$
\subsection{Lien aux primitives}
Soit $F_a$ la fonction telle que $F_a(x) = \int_{a}^{x}f(x)\,dx$, $F_a$ est la primitive de $f$ qui s'annule de $a$. En effet, $\int_{a}^{a}f(x)\,dx = 0$ et $F_a'(x) \in \lim \limits_{h \rightarrow 0}[f(x);f(x+h)]$, donc $F_a'=f$.
\subsubsection{Application directe}
Ainsi, $\int_{a}^{b}f(x)\,dx = F_a(b) = F_a(b) - F_a(a) = (F_a(b)+k) - (F_a(a)+k)\, \forall k\in\mathbb{R} = F(b)-F(a) = [F(x)]_a^b$ avec $F(x) = F_a(x)+k \, \forall k\in\mathbb{R}$, soit une quelconque primitive de $f$.
\subsection{Propriétés}
Avec $f$, $g$ des fonctions continues sur $[a;b]$, $F$ et $G$ leurs primitives, $k$ un réel, et $c\in[a;b]$
\begin{enumerate}
	\item (Linéarité) $\int_{a}^{b}kf(x)\\ = k\int_{a}^{b}f(x)\,dx$
	\item (Linéarité) $\int_{a}^{b}f(x)+g(x)\,dx = \int_{a}^{b}f(x)\,dx+\int_{a}^{b}g(x),dx$
	\item (Relation de Chasles) $\int_{a}^{b}f(x)\,dx = \int_{a}^{c}f(x)\,dx + \int_{c}^{b}f(x)\,dx$
	\item (Notable:) $\int_{a}^{b}f(x)g(x)\,dx = [f(x)G(x)]_a^b- \int_{a}^{b}f'(x)+G(x)\,dx$
\end{enumerate}

\section{Graphes}
\begin{enumerate}
	\item Un graphe est un ensemble de sommets reliés par des arrêtes, orientée ou non.
	\item L'ordre d'un graphe est son nombre de sommets.
	\item Deux sommets reliés sont dits adjacents.
	\item Le degré d'un sommet est le nombre d'arètes dont ce sommet est une extrémité (un boucle comptant deux fois).
	\item Un graphe est dit complet lorsque tout ses sommets sont adjacents.
\end{enumerate}
\subsection{Chaîne}
\begin{enumerate}
	\item Une chaîne est un parcours dans un graphe ex: $A-B-C-D-B-F$
	\item Elle est dite fermée quand l'origine et l'extrémité sont confondues ex: $A-B-C-D-B-A$
	\item Un cycle est une chaîne fermée dont toutes les arrêtes sont distinctes ex: $B-C-D-B$
\end{enumerate}
\subsection{Matrice d'adjacence}
La matrice d'adjacence $\mathbf{M}$ associée à un graphe d'ordre $n$ dont les sommets sont numérotés de $1$ à $n$ est la matrice carrée d'ordre $n$ telle que le coefficient $\mathbf{m}_{i,j}$ de la matrice $M^p$ vaut le nombre de chemins de longueur $p$ reliant les sommets $i$ et $j$, soit le nombre d'arètes entre $i$ et $j$ quand $p=1$




\pagebreak\part{Géométrie}
\setcounter{section}{0}
\renewcommand*{\theHsection}{chX.\the\value{section}}


\section{Vecteurs}
\href{https://youtu.be/fNk_zzaMoSs}{\underline{\textit{Vidéo (anglais)}}}\\
\break

Le vecteur $\Vec{AB}$ se note $(x_b - x_a ; y_b - y_a ; ...)$.\\
$I$ est milieu de $AB$ ssi $I=(\frac{x_a+x_b}{2};\frac{y_a+y_b}{2};...)$



\subsubsection{Norme d'un vecteur}
On note la norme $AB$ du vecteur $\Vec{AB}$ : $\norm{\Vec{AB}}$.\\
Dans un repère orthonormé (euclidien), $\norm{\Vec{AB}}$ peut se calculer avec la formule de la distance euclidienne, comme suit: $\sqrt{(x_B-x_A)^2+ (y_B-y_A)^2+...}$.

\subsubsection{Vecteur opposé}
Le vecteur opposé a $\Vec{AB}$ est $-\Vec{AB} = \Vec{BA}$.

\subsubsection{Relation de Chasles}
Les vecteurs ne prennent pas compte des détours. Ainsi, $\Vec{AB}+\Vec{BC}=\Vec{AC}$.

\subsubsection{Colinéarité}
Si $\Vec{v} = k\Vec{u}$ avec $k$ un réel, alors $\Vec{v}$ et $\Vec{u}$ sont colinéaires. Dès lors, $\Vec{v}_x\Vec{u}_y - \Vec{u}_x\Vec{v}_y = 0$

\subsection{Coplanairité}
Si il existe des réels $\alpha,\beta\in\mathbb{R}\mid\Vec{w}=\alpha\Vec{u}+\beta\Vec{v}$ alors $\Vec{u},\Vec{v},\Vec{w}$ sont coplanaires.\\
De même, il sont coplanaires ssi il existe $a,b,c\in\mathbf{R^*}$ tels que $a\Vec{u}+b\Vec{v}+c\Vec{w}=0$. Par contraposée, si ces trois coefficients sont nuls alors les vecteurs ne sont pas coplanaires.

\subsection{Produit scalaire}
\href{https://youtu.be/LyGKycYT2v0}{\underline{\textit{Compréhension la dualité entre le produit scalaire et les produits de matrices (anglais)}}}\\
\break
Le produit scalaire, de deux vecteurs se note $\Vec{AB} \cdot \Vec{AC}$.\\
\\
Méthodes de calcul:
\begin{enumerate}
	\item Par le projeté orthogonal \\Soit $H$ le projeté orthogonal de $C$ sur $(AB)$: $\Vec{AB} \cdot \Vec{AC} = \Vec{AB} \cdot \Vec{AH} = AB \times AH$.
	\item Par les coordonnées \\$\Vec{AB} \cdot \Vec{AC} = x_{AB}x_{AC}+y_{AB}y_{AC}$
	\item Par le cosinus \\$\Vec{AB} \cdot \Vec{AC} = AB\times AC\times \cos(\widehat{BAC})$
	\item Par les normes \\$\Vec{AB} \cdot \Vec{AC} = \frac{1}{2}(\lVert \Vec{AB} \rVert^2 + \lVert \Vec{AC}\rVert^2 - \rVert \Vec{AB} - \Vec{AC} \rVert ^2) \\ = \frac{1}{2}(\lVert \Vec{AB} + \Vec{AC} \rVert ^2 - \lVert \Vec{AB} \rVert^2 - \lVert \Vec{AC}\rVert^2)$
\end{enumerate}
On notera que $\Vec{AB}$ et $\Vec{AC}$ sont orthogonaux si leur produit scalaire est nul.


\section{Plan}
\href{https://youtu.be/k7RM-ot2NWY}{\underline{\textit{Vidéo (anglais)}}}\\
\break
Un plan est défini dans l'espace par deux vecteurs non colinéaires et un point.\\
Dès lors pour un plan $\mathcal{P}$ défini par le point $A$ et les vecteurs $\Vec{u}$ et $\Vec{v}$, $M \in \mathcal{P}$ ssi $\Vec{AM} = a\Vec{u}+b\Vec{v}$ avec $a,b\in\mathbb{R}$
\subsection{Vecteur normal}
Est appelé vecteur normal d'un plan le vecteur orthogonal à tout les vecteurs du plan, soit à ses deux vecteurs directeurs. Dès lors, toute droite ayant ce vecteur comme vecteur directeur est orthogonale au plan.

\subsection{Équation cartésienne}
L'équation cartésienne d'un plan est l'équation $ax+by+cz+d = 0$ telles que tout les points $(x;y;z)$ du plan vérifient cette équation. On notera que le vecteur $\Vec{\begin{pmatrix}a\\b\\c\end{pmatrix}})$ est le vecteur normal du plan.

\section{Distance d'un point à ...}
\paragraph{Un point}
Pour trouver la distance $AB$, il suffit de trouver la norme du vecteur $\Vec{AB}$.
\paragraph{Un plan}
La distance d'un point à un plan est la distance de ce point au projeté orthogonal de ce point sur ce plan. Ces deux points forment un vecteur dont on peut chercher la normes, et qui a la particularité d'être normal du plan, donc colinéaire à tout vecteur normal de ce plan.


\section{Base de l'espace}
Trois vecteurs non coplanaires forment une base de l'espace\\
Dès lors, $\forall \Vec{u} \in$ la base de l'espace $(\Vec{i};\Vec{j};\Vec{k})$, $\exists x,y,z \in \mathbb{R}$ tels que $x\Vec{u}+y\Vec{j}+z\Vec{k}=\Vec{u}$. On dira que $\Vec{u} = \begin{pmatrix}x\\y\\z\end{pmatrix}$ dans $(\Vec{i};\Vec{j};\Vec{k})$


\section{Formes géométriques}

\subsection{En 2D}
\subsubsection{Droite}
Une équation de droite en 2D peut s'écrire de deux manières: la forme réduite $y=mx+p$, et la forme cartésienne $ay+bx+c=0$.\\
On distingue alors le vecteur directeur et le vecteur normal (orthogonal au premier), de formules respectives $\begin{pmatrix}-b\\a\end{pmatrix}$ et $\begin{pmatrix}a\\b\end{pmatrix}$.
\subsubsection{Cercle}
Ici la forme cartésienne est $(x-x_A)^2+(y-y_A)^2 = r^2$ avec $A$ le centre du cercle et $r$ son rayon.\\
Dès lors, l'ensemble des les points $(x;y)$ qui vérifient $(x-a)^2+(y-b)^2 = S$ avec $S\in\mathbb{R^+}$, est un cercle de centre $(a;b)$ et de rayon $\sqrt{S}$.
\subsubsection{Parabole}
On retrouve la forme $y=ax^2+bx+c$ où on remarque l'axe de symétrie $x = -b/2a$ et le sommet $(-b/2a,f(-b/2a))$.

\subsection{En 3D}
\subsubsection{Droite}
Soit $(\Vec{i};\Vec{j};\Vec{k})$ une base de l'espace et $D$ une droite $(x_a;y_a;z_a;\Vec{\begin{pmatrix}a\\b\\c\end{pmatrix}})$,\\
Les points de $D$ suivent le mode de génération $(x;y;z)$ suivant:
\[
	\left \{
	\begin{array}{c @{=} c}
		x & x_a + ak \\
		y & y_a + bk \\
		z & z_a + ck 
	\end{array}
	\forall k \in \mathbb{R}
	\right.
\]


\section{Transformations}
Dans un repère $(\mathbf{0},\Vec{i},\Vec{j})$, on considère les points $\mathbf{A}(x_a;y_a)$ et $\mathbf{B}(x_b;y_b)$ et un vecteur $\Vec{u}\begin{pmatrix}a\\b\end{pmatrix}$

\subsection{Translation}
$\mathbf{B}$ est l'image de $\mathbf{A}$ par translation de vecteur $\Vec{u}$ ssi $\begin{pmatrix}x_b\\y_b\end{pmatrix} = \begin{pmatrix}x_a\\y_a\end{pmatrix} + \begin{pmatrix}a\\b\end{pmatrix}.$

\subsection{Rotation}
$\mathbf{B}$ est l'image de $\mathbf{A}$ par rotation de centre $\mathbf{0}$ et d'angle $\theta$ ssi $$\begin{pmatrix}x_b\\y_b\end{pmatrix} = \begin{pmatrix}\cos \theta & -\sin \theta \\\sin \theta & \cos \theta\end{pmatrix} \times \begin{pmatrix}x_a\\y_a\end{pmatrix}$$



\pagebreak\pagebreak\part{Statistiques et Probabilités}
\setcounter{section}{0}
\renewcommand*{\theHsection}{chX.\the\value{section}}


\section{Vocabulaire}
Soit deux évènements $\mathcal{A}$ et $\mathcal{B}$ :\\
On note $\mathcal{A} \cup \mathcal{B}$ l'union des deux évènements, soit celui où au moins l'un des deux se produit.\\
On note $\mathcal{A} \cap \mathcal{B}$ l'intersection des deux évènements, soit celui où les deux se produisent.\\
On note $\mathcal{_AB}$ l'évènement ou $\mathcal{B}$ se produit, sachant que $\mathcal{A}$ s'est déjà produit (utilise typiquement dans les arbres pondérés).\\


\section{Lois et Formules}
$$\mathcal{P_A} \cup \mathcal{P_B} = \mathcal{P_A} +\mathcal{P_B} - \mathcal{P_A} \cap \mathcal{P_B}$$
$$\mathcal{P_A} \cap \mathcal{P_B} = \mathcal{P_A} +\mathcal{P_B} - \mathcal{P_A} \cup \mathcal{P_B}$$
$$\mathcal{P_AB} = \frac{\mathcal{PA} \cap \mathcal{PB}}{\mathcal{PA}}$$
$$\mathcal{P_BA} = \frac{\mathcal{P_AB} \times \mathcal{PA}}{\mathcal{PB}}$$
\break
\href{https://youtube.com/playlist?list=PLj2YHanZadz49HJPJqMucrW_KgTVXcmmP}{\underline{\textit{Vidéos sur le théorème de Bayes, soit la dernière formule ci dessus (anglais)}}}
\subsection{Loi des probabilités totales}
La probabilité d'un événement est égale à la somme des probabilités des chemins conduisant à cet événement.


\section{Indépendance}
$\mathcal{A}$ et $\mathcal{B}$ sont indépendants si et seulement si $\mathcal{PA} \cap \mathcal{PB} = \mathcal{PA} \times \mathcal{PB}$. \\
Des lors, $\mathcal{P_BA} = \mathcal{PA}$ et $\mathcal{P_AB} = \mathcal{PB}$.



\section{Dénombrement}
On note $\mathbf{Card(E)}$ le nombre d'éléments dans l'ensemble $\mathbf{E}$.

\subsection{Multiplication ou addition}
Si deux évènements sont liés, on appliquera une multiplication, si ils sont successifs, une addition.

\subsection{Possibilités ordonnées}
On fait le produit cartésien des ensembles des possibles pour obtenir les possibilités ordonnées, soit l'ensemble des n-uplets  de chaque objet avec chaque autre.
\underline{Exemple:} soit deux ensembles $A = \{1;2\}$ et $B=\{a;b\}$.  Alors $A\times B = \{(1;a);(1;b);(2;a);(2;b)\}$.

\subsection{Arrangements}
$$\frac{n!}{(n-k)!}$$ est le nombre d'arrangements possibles de $k$ éléments parmi $n$

\subsection{Permutations}
$$n!=\prod_{k=0}^nk$$ est le nombre des permutation possibles pour $n$ éléments

\subsection{Binôme}
$$\frac{n!}{k!(n - k)!} = \binom{n}{k}$$ est le nombre de façons de prendre $k$ éléments depuis $n$.
\subsubsection{Pascal}
$$\binom{n+1}{k} = \binom{n}{k} + \binom{n}{k-1}$$
\break
\paragraph{Exemple d'application}
Le nombre de cas avec au moins un objet est la différence entre le nombre de cas possibles et le nombres de cas sans cet objet: $\binom{n-1}{k-1} = \binom{n}{k} - \binom{n-1}{k}$



\section{Variables aléatoires}

Une variable aléatoires est une variable pouvant prendre différentes valeurs, suivant une certaine probabilité. \\
On associe alors à la variable aléatoire $X$ les valeurs $x_1, x_2, ..., x_n$ et leurs probabilités respectives $p_i$ définies par $P(X = x_i)$.


\subsection{Paramètres}
Une variable aléatoire a dès lors des paramètres, définis comme suis:
\begin{enumerate}
	\item L'espérance, $E(X)$, qui représente la moyenne pondérée des valeurs possibles de cette variable, et qui est définie par: $$\sum^n_{i=1}p_ix_i$$
	\item Sa variance, une étape de calcul de son écart-type, $V(X)$: $$\sum^n_{i=1}p_i(x_i-E(X))^2$$
	\item Son écart-type, soit la distance moyenne d'un tirage à l'espérance, $\sigma(X)$: $$\sqrt{V(X)}$$
\end{enumerate}

\subsection{Somme de variable aléatoires}
Soit $X$ et $Y$ des variables aléatoires, et $a\in\mathbb{R}$.\\
On définis $X + Y$ comme la somme des valeurs de $X$ et de $Y$, et $aX$ comme le produit de $a$ et de la valeur de $X$.
On a:
\begin{enumerate}
	\item $E(X + Y) = E(X) + E(Y)$ (Linéarité de l'espérance)
	\item $E(aX) = a\times E(X)$ (Linéarité de l'espérance)
	\item[]
	\item \emph{Ssi $X$ et $Y$ sont indépendantes}: $V(X + Y) = V(X) + V(Y)$ 
	\item $V(aX) = a^2V(X)$
	\item $\sigma(aX) = \abs{a}\sigma(X)$
\end{enumerate}

\subsection{Échantillons de variables aléatoires}
Un échantillon de taille $n\in\mathbb{N}$ associée a une loi est une liste de $n$ variables aléatoires \emph{indépendantes et suivant cette loi}.
\subsubsection{Somme d'un échantillon}
Soit $S_n$ la somme d'un échantillon longueur $n$ associé à la loi d'une variable aléatoire $X$. On a:
\begin{enumerate}
	\item $E(S_n) = nE(X)$
	\item $V(S_n) = nV(X)$
	\item $\sigma(S_n) = \sqrt{n}\sigma(X)$
\end{enumerate}
\subsubsection{Moyenne d'un échantillon }
Soit $M_n = \frac{1}{n}S_n$. On applique les règles sur la somme de variables aléatoires pour obtenir:$E(S_n) = E(X)$, $V(S_n) = V(X)/n$ et $\sigma(S_n) = \sigma(X)/\sqrt{n}$.

\subsection{Expérience à 2 issues}
Soit une expérience de probabilité de succès $p$.\\ La variable aléatoire correspondante prendra comme valeur $1$ en cas de succès, $0$ en cas d'échec.
\\ Cette expérience est appelée \textit{Épreuve de Bernoulli}
\subsubsection{Loi de Bernoulli}
La variable $X$ qui rend compte du succès ou de l'échec d'une épreuve de Bernoulli suit une loi de Bernoulli de paramètre $p$. \\
On note $X \sim \mathcal{B}(p)$. \\
Elle aurais pour propriétés les suivantes: Espérance= $p$, Variance= $p(1-p)$, Écart-type= $\sqrt{p(1-p)}$.
\subsubsection{Loi binomiale}
Si on répète une expérience de Bernoulli de manière indépendante et identique $n$ fois, on trouve un schéma de Bernoulli.\\
La variable $X$ qui compterait le nombre de succès suis une loi binomiale de paramètres $n$ et $p$. \\
On note $X \sim \mathcal{B}(n;p)$. \\
Elle aurais pour propriétés les suivantes: Espérance= $np$, Variance= $np(1-p)$, Écart-type= $\sqrt{np(1-p)}$\\
On peut calculer la probabilité de $k$ succès telle que $$\mathcal{P}(X=k)= \binom{n}{k}\times p^n\times (1-p)^{n-k}$$.
\subsection{Intervalle de fluctuation}
Soit $X$ une variable aléatoire et $a,b,\alpha \in \mathbb{R}$ \\
Une intervalle $[a;b]$, telle que $\mathcal{P}(a\le X \le b) > 1-\alpha$, est une intervalle de fluctuation au seuil de $1-\alpha$ (au risque de $\alpha$). Elle vaut typiquement $\sum_a^b p_i$.


\pagebreak\pagebreak
\subsection{Concentration}
\subsubsection{Inégalité de Bienaymé-Tchebychev et de concentration}
$\forall \delta > 0$
$$P(\abs{X - E(X)} \ge \delta) \le \frac{V(X)}{\delta^2}$$
Cette inégalité peut être appliquée à une moyenne:
$$P(\abs{M_n - E(X)} \ge \delta) \le \frac{V(X)}{n\delta^2}$$
\subsubsection{Loi faible des grands nombres}
$$\lim \limits_{n \rightarrow +\infty} P(\abs{M_n - E(X)} \ge \delta) = 0$$

\part{Arithmétique}
\setcounter{section}{0}
\renewcommand*{\theHsection}{chX.\the\value{section}}


\section{Divisons}

\subsection{Divisibilité}
Propriétés (On travaille dans $\mathbb{Z}$):
\begin{enumerate}
	\item $b|a$ ($b$ divise $a$) ssi $a=kb$
	\item $a|b$ et $b|c \Rightarrow a|c$\\
	      Démonstration: $b=ka$ et $c=k'b$ donc $c=k'ka$ or $k'k\in\mathbb{Z}$ donc $a|c$
	\item $c|a$ et $c|b \Rightarrow c|(au+bv)$\\
	      Démonstration: $a=kc$ et $b=k'c$ donc $au+bv=kcu+k'cv=c(ku+k'v)$ or $(ku+k'v)\in\mathbb{Z}$ donc $c|(au+bv)$
\end{enumerate}
Critères de divisibilité:
\begin{enumerate}
	\item Est divisible par 2 ou 5 si son chiffre des unités est respectivement divisible par 2 ou 5.
	\item Est divisible par 4 ou 25 si son nombre (dizaines:unités) est respectivement divisible par 4 ou 25.
	\item Est divisible par 3 ou 9 si la somme de ses chiffres est respectivement divisible par 3 ou 9.
	\item Est divisible par 11 si la somme de ses chiffres de rang impair moins la somme de ses chiffres de rang pair est divisible par 11.
\end{enumerate}

\subsection{Division euclidienne}
Soit $a,b\in\mathbf{N^*}$, il existe un unique couple d'entiers $(q;r)$ tel que $a=qb+r$ où $0\le r < b$. \\Cette écriture est la division euclidienne de $a$ par $b$, avec $a$ le dividende, $b$ le diviseur, $q$ le quotient et $r$ le reste.

\subsection{Congruence}
$a$ et $b$ sont dits congrus modulo $n$ si $n|(a-b)$, soit que $r_{a/n}=r_{b/n}$. On note $a\equiv b (n)$
\\\\Propriétés;
\begin{enumerate}
	\item $a \equiv b(n)\,b\equiv c(n) \rightarrow a\equiv c(n)$
	\item $a \equiv b(n)\,c \equiv d(n) \rightarrow a+c\equiv b+d(n)$
	\item $a \equiv b(n)\,c \equiv d(n) \rightarrow ac\equiv bd(n)$
	\item $a \equiv b(n)\,k\in\mathbb{Z} \rightarrow ak\equiv bk(n)$
	\item $a \equiv b(n)\,p\in\mathbb{N} \rightarrow a^p\equiv b^p(n)$
\end{enumerate}


\section{Matrices}
\href{https://youtu.be/kYB8IZa5AuEl}{\underline{\textit{Vidéo (anglais)}}}\\
\break
\begin{enumerate}
	\item Une matrice est un tableau de dimension lignes$\times$colonnes rempli de termes/coefficients.
	\item Une matrice ligne est une matrice dotée d'une unique ligne, et inversement pour une matrice colonne.
	\item Deux matrices sont égales si elles sont de même taille et de mêmes termes.
	\item Une matrice carrée d'ordre $p$ est une matrice de dimensions $p\times p$.
	\item Une matrice diagonale a tout les termes hors de sa diagonale (haut-gauche $\rightarrow$ bas-droite) de nuls.
	\item Une matrice est triangle supérieure ou inférieure si tout ses termes respectivement inférieurs ou supérieurs à sa diagonale sont nuls
	\item On note $I_p$ la matrice carrée d'ordre $p$ diagonale, dont les termes de la diagonale valent $1$.
	\item La transposée de la matrice $A$ de dimensions $n\times p$, notée $^TA$ (cette notation peut varier) de dimensions $p\times n$, obtenue en échangeant les colonnes et les lignes de $A$.
\end{enumerate}
\underline{Note:} Une matrice peut être visualisée comme une transformation de l'espace (cf. Vidéo ci-dessus).

\subsection{Déterminant ($det(A)$)}
\href{https://youtu.be/Ip3X9LOh2dk}{\underline{\textit{Vidéo (anglais})}}\\
\href{https://youtu.be/v8VSDg_WQlA}{\underline{\textit{Compréhension des matrices non carrées et de pourquoi elles n'ont pas de déterminant (anglais)}}}\\
\break
Le déterminant d'une matrice correspond au coefficient par lequel toute aire (pour un espace de dimension 2) ou volume (dimension 3), etc... d'un espace est multiplié lors de l'application de la matrice aux vecteurs-unités.\\
Ainsi, un déterminant nul signifie que l'espace est réduit à un espace de dimension inférieure.\\
\underline{Note:} Un déterminant négatif représente une sorte d'inversion des directions.
\subsubsection{Pour une matrice carré d'ordre 2}
On peut calculer le déterminant d'une matrice carré d'ordre 2 de la forme
$ A = \begin{pmatrix}
a & b \\
c & d
\end{pmatrix}$, on peut calculer son déterminant tel que $det(A)=ad-bc$
\subsubsection{Pour une matrice carré d'ordre $n$}
On peut calculer le déterminant d'une matrice carré $A$ d'ordre $n$ comme suit:
$$det(A) = \sum_{i=1}^n (-1)^{i+j} \times det(A\Delta_{i,j}) \mid 1\le j \le n = \sum_{j=1}^n (-1)^{i+j} \times det(A\Delta_{i,j}) \mid 1\le i \le n$$
Avec $A\Delta_{i,j}$ la matrice $A$ privée de sa ligne $i$ et de sa colonne $j$, et $a_{i,j}$ son terme en ligne $i$ et colonne $j$.

\subsection{Opérations}
\subsubsection{Additions}
\begin{enumerate}
	\item $A+B = a_{i,j}+b_{i,j}$
	\item $kA = ka_{i,j}$
	\item Cette addition est commutative.
	\item Les règles de distributivité et de factorisation s'appliquent.
\end{enumerate}
\hfill \break
\subsubsection{Multiplication}
\href{https://youtu.be/XkY2DOUCWMU}{\underline{\textit{Vidéo (anglais)}}}\\
\break
\begin{enumerate}
	\item Pour effectuer le produit de $A$ par $B$, $A$ doit avoir autant de lignes que $B$ de colonnes.
	\item La matrice obtenue par le produit d'une matrice de taille $n \times p$ et une autre de $p \times q$ est de taille $n \times q$.
	\item $$A\times B=\sum_{k=1}^{p} a_{i,k} \times b_{k,j}$$
	\item Ce produit n'est pas commutatif.
	\item $A(n\times n) \times I_n = A$
\end{enumerate}
\subsubsection{Puissances}
\begin{enumerate}
	\item $A^2 = AA$
	\item $A^0 = I$
\end{enumerate}
\subsubsection{Inverse ($A^{-1}$)}:\\
Une matrice est $A$ dite inversible ssi $\exists B, AB = BA = I$. On note alors $B = A^{-1}$. En somme, il s'agit d'appliquer la transformation inverse, et donc de revenir à notre état d'origine. Une matrice est inversible ssi $det(A) \ne 0$, soit que cette transformation ne nous a pas envoyé dans un espace de dimension inférieure, dans lequel cas il serait impossible de revenir au cas précédent, l'opération inverse d'une multiplication par $0$ étant indéfinie (division par $0$).
\break
Dans le cas d'une matrice carré d'ordre 2 de la forme
$\begin{pmatrix}
	a & b \\
	c & d 
\end{pmatrix}$
, $A^{-1} = \frac{1}{det(A)}
\begin{pmatrix}
	d  & -b \\
	-c & a  
\end{pmatrix}$.

\subsection{Caractéristiques d'une matrice}
\href{https://youtu.be/PFDu9oVAE-g}{\underline{\textit{Vidéo sur les vecteurs et valeurs propres (anglais)}}}\\
\break
Soit $A$ une matrice carrée, qui représentera toutes les matrices carrées pour ce qui suit.
\subsubsection{Vecteurs propres}
Un vecteur propre d'une matrice est un vecteur non nul qui ne change pas de sens ni de direction, mais dont la normes est multipliée par un certain facteur, soit que $A\Vec{v} = \lambda \Vec{v}$.
\subsubsection{Valeurs propres}
Les valeurs propres d'une matrice sont les valeurs $\lambda$ pour lesquelles il existe $\Vec{v} \ne \Vec{0}$ tel que $A\Vec{v} = \lambda \Vec{v}$.\\
\underline{Note:}
\begin{impl}
	\item[] $A\Vec{v} = \lambda \Vec{v}$
	\item $A\Vec{v} - \lambda \Vec{v} = 0$
	\item $(A - \lambda I)\Vec{v} = 0$
	\item[$\Rightarrow$] $det(A - \lambda I) = 0$ car $\Vec{v} \ne 0$, et que donc le seul moyen pour qu'un vecteur soit transformé en vecteur nul est que l'espace soit ramené à un espace de dimension inférieure.
\end{impl}
\break
\href{https://youtu.be/e50Bj7jn9IQ}{\underline{\textit{Vidéo d'une technique de calcul rapide des valeurs propres (anglais)}}}
\subsubsection{Polynôme caractéristique}
Le polynôme caractéristique d'une matrice est, justement le polynôme tel que $P(\lambda) = det(A - \lambda I)$. Ses racines seront alors les valeurs propres de la matrice, et alors il suffira de trouver les $\Vec{v}$ tels que $(A - \lambda I)\Vec{v} = 0$.
\subsubsection{Espaces propres}
L'espace propres $E_\lambda$ d'une matrice est l'ensemble tel que $$E_\lambda = \{ \Vec{v} \mid (A - \lambda I)\Vec{v} = 0 \}$$.

\subsection{Systèmes grâce aux matrices}
On peut écrire un système sous la forme de produit de la matrice $\mathbf{A}$ qui contient les coefficients, la matrice $\mathbf{X}$ qui contient les inconnues, et la matrice $\mathbf{B}$ qui contient les résultats. On trouve alors $\mathbf{AX} = \mathbf{B}$, donc $\mathbf{X} = \mathbf{A}^{-1}\mathbf{B}$.\\
\break
\href{https://youtu.be/jBsC34PxzoM}{\underline{\textit{Vidéo sur une méthode alternative, méthode de Cramer (anglais)}}}

\subsection{Suites de matrices}
De même, ont peut écrire un système de génération de suites sous la forme d'une équation de matrice. Un exemple d'application serait: Admettons une relation de récurrence $\mathbf{X}_{n+1} = \mathbf{AX}_n+\mathbf{B}$, on peut trouver la limite $\mathbf{X}$ telle que $\mathbf{X} = \mathbf{AX}+\mathbf{B}$ en utilisant l'inverse de $(\mathbf{I}-\mathbf{A})$.


\section{Nombre complexes}
Il existe un ensemble, celui des nombres complexes ($\mathbb{C}$), qui comprend et fonctionne comme $\mathbb{R}$, à l'exception qu'il comporte un élément noté $i$ défini tel que $i^2 = -1$. Tout élément de cet ensemble (notés $z$ par convention) s'écrie dès lors sous la forme $z=a+bi$ avec $a,b\in\mathbb{R}$, où $a$ est sa partie réelle ($Re(z)$) et $b$ sa partie imaginaire ($Im(z)$).\\
Un imaginaire pûr a une partie réelle nulle et un réel pûr a une partie imaginaire nulle.

\subsection{Conjugué}
On note $\bar{z}$ le nombre complexe tel que si $z=a+bi$, $\bar{z}=a-bi$ ($a,b\in\mathbb{R}$).\\
\paragraph{Conséquences}
\begin{enumerate}
	\item $z+\bar{z} = 2Re(a)$ donc $z+\bar{z}=0 \Leftrightarrow  Re(z)=0$
	\item $z-\bar{z} = 2iIm(a)$ donc $z-\bar{z}=0 \Leftrightarrow z=\bar{z} \Leftrightarrow Im(z)=0$
	\item $z\times\bar{z} = Re(z)^2 + Im(z)^2$ (cf 4ème identité remarquable)
\end{enumerate}


\pagebreak\part{Boite à outils}
\setcounter{section}{0}
\renewcommand*{\theHsection}{chX.\the\value{section}}



\section{Règles sur les inégalités}
\begin{enumerate}
	\item Les additions et soustraction \emph{ne changent pas le sens}.
	\item Les multiplication et division \emph{ne changent le sens QUE SI leur produit ou dividende est négatif}.
	\item L'application d'une fonction \emph{ne change le sens QUE SI sa fonction est décroissante et continue sur l'intervalle des termes} et \emph{ne change PAS le sens QUE SI sa fonction est strictement croissante et continue sur l'intervalle des termes}.
\end{enumerate}


\section{Prouver une proposition}

\subsection{Par déduction}
On part d'une hypothèse et par un raisonnement logique, on arrive à une conclusion.

\subsection{Par contre exemple}
On trouve un cas qui contredit la proposition selon laquelle la proposition que l'on essaye de prouver est fausse.

\subsection{Par l'absurde}
On suppose que la proposition est fausse et arrive à une contradiction.

\subsection{Par contraposée}
On prouve la contraposée pour prouvée la proposition

\subsection{Par disjonction de cas}
Lorsque la proposition dépend d'un variable, on sépare le raisonnement suivant les valeurs de la dite variable.

\subsection{Par récurrence}
On prouve qu'une proposition est vrai à partir d'un certain rang, en prouvant qu'elle est initialisée pour ce rang puis qu'elle est héréditaire.\\
\subsubsection{Exemple de la rédaction conseillée}
Soit la suite $u$ telle que \[
\left \{
\begin{array}{c @{=} c}
	u_0     & 2      \\
	u_{n+1} & 5u_n+4 
\end{array}
\right.
\]
Montrons par récurrence que $\forall n\in\mathbb{N}, u_n>0$\\
Pour tout entier $n\in\mathbb{N}$, notons $\mathbf{P(n)}: \text{"}u_n>0\text{"}$\\
\break
\underline{Initialisation pour n=0}\\
$u_n=u_0=2$ or $2>0$ donc $u_0>0$.\\
$\mathbf{P(0)}$ est ainsi, donc $\mathbf{P}$ est donc initialisée.\\
\break
\underline{Hérédité}\\
Supposons $n\in\mathbb{N}$ telle que $\mathbf{P(n)}$ vraie, montrons que $\mathbf{P(n+1)}$ est vraie.\\
$u_n>0$ donc $5u_n+4>0$ donc $u_{n+1}>0$ donc $\mathbf{P(n+1)}$ est vraie.\\
Ainsi $\mathbf{P(n)} \implies \mathbf{P(n+1)}$.\\
\break
\underline{Conclusion}\\
$\mathbf{P(n)}$ est initialisée et héréditaire, donc par principe de récurrence, $\forall n\in\mathbb{N}, \mathbf{P(n)}$ est vraie soit $\forall n\in\mathbb{N}, u_n>0$.

\section{Factorisation}

\subsection{Identités remarquables}
\begin{itemize}
	\item $(a+b)^2 = a^2 + 2ab + b^2$
	\item $(a-b)^2 = a^2 - 2ab + b^2$
	\item $a^2 - b^2 = (a+b)(a-b)$
	\item $a^2 + b^2 = (a+bi)(a-bi)$ (dans $\mathbb{C}$)
\end{itemize}
\subsubsection{Formule du binôme de Newton}

De manière générale, on peut écrire:
$$(a+b)^n = \sum^n_{k=0}\binom{n}{k}a^{n-k}b^k$$

\underline{Note:} On a alors $(a+b)^n = (P_0a^n + P_1a^{n-1}b + ... + P_{n-1}ab^{n-1} + P_nb^n)$ avec $P_k$ la $k$ième valeur de la $n$ième ligne du triangle de Pascal.

\subsection{Polynôme}
Une fonction polynôme $P$ de degré $n$ est de la forme $P(z) = \sum^n_{k=0}a_kz^n$ avec $a_k$ un coefficient complexe.\\
\subsubsection{Théorèmes}
\begin{enumerate}
	\item Soit $n\in\mathbb{Z}, a\in\mathbb{C}$, il existe un polynôme $Q$ de degré $n-1$ tel que $a^n-b^n = (a-b)Q(z)$. (Note: $Q(z) = \sum_{k=0}^{n-1} a^{n-1-k}\times b^k$)
	\item Soit $P$ un polynôme de degré $n$ et $a\in\mathbb{C} \mid P(a)=0$, il existe un polynôme $Q$ de degré $n-1$ tel que $P(z) = (z-a)Q(z)$.
	\item Soit $P$ un polynôme de degré $n$, il a $n$ racines ou moins.
\end{enumerate}





\end{document}